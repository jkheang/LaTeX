\documentclass[12pt]{article}
\usepackage[utf8]{inputenc}
\usepackage{amsmath}
\usepackage{tikz}
\usepackage{comment}
\usepackage{amssymb}


\title{Technical Report}
\author{Jacky Kheang\\Jaylie Arelana Galang\\Andres Zazueta\\Elizabeth Brown\\Joseph Chin}
\date{8th of May, 2019}
\begin{document}
\begin{titlepage}
\maketitle
\end{titlepage}

Here is the diagram we are tasked with analyzing:
\newline

\begin{figure}[h!]
	\includegraphics[width=\linewidth]{FBD3.png}
\end{figure}
From this diagram, we will calculate stress at point A and the total strain of the bar when:\\
\indent$1) \Delta = 0.3 mm$\\	
\indent$2) \Delta = 0.2 mm$\\ 
\indent$3) \Delta = 0.1 mm$\\

The bar experiences a temperature increase of 20ºC. We are also given:\\
$E=100 GPa\\
\nu=0.25\\
\alpha=\frac{10^{-5}}{Cº}$
\pagebreak

Some necessary assumptions to make before we analyze the free body diagram is that there is no thermal stress associated with thermal strain. Thermal stress is a result of the loading condition. In this particular situation, a statically indeterminate body will result in thermal stress. The necessary equations and relations we employed will be listed below and referred to throughout the documentation.
\newline

Stress due to load:
\begin{equation} \label{stress}
\sigma=\frac{P}{A}
\end{equation}
Where $P$ is the applied load and $A$ is the cross sectional of a homogeneous body.
\newline

Strain due to deformation:
\begin{equation} \label{strain}
\epsilon=\frac{\delta}{L_{º}}
\end{equation}
Where $\delta$ is the total deformation and $L_{º}$ is the original length of the body.
\newline

Deformation due to loading:
\begin{equation} \label{deform_load}
\delta_{load} = \frac{PL}{EA}
\end{equation}
Where $E$ is Young's Modulus which is a material property. Since the body is homogeneous, $E$ is constant. In order to find $L$, we must consider the following conditions:
\begin{enumerate}
	\item The internal loading must be constant
	\item The cross-sectional area must be constant
	\item $E$, Young's Modulus is constant
\end{enumerate}
$\therefore L$ is the corresponding length during which the 3 other parameters are constant.
\newline

Deformation due to thermal expansion:
\begin{equation} \label{deform_thermal}
\delta_{thermal} = \alpha \cdot \Delta T \cdot L
\end{equation}
$\alpha$ is the thermal expansion coefficient, $\Delta T$ is the change in temperature, and $L$ is the length of the body.
\pagebreak

Poissons's Ratio:
\begin{equation} \label{poisson}
\nu=\frac{-\epsilon_{y}}{\epsilon_{x}}
\end{equation}
Poisson's ratio is a material property that is given but will be important because during thermal expansion, deformation occurs in multiple directions.
\newline

Stress-Strain relationship:\\
We can come to this relationship by looking at equation \ref{strain} and substituting $\delta$ with \ref{deform_load}.
\newline
\begin{center}
$\epsilon = \frac{\delta}{L_{º}} = \frac{\frac{PL}{EA}}{L_{º}}$
\end{center}
$L$ cancels, which will give:
\newline
\begin{center}
$\epsilon = \frac{P}{A}E = \sigma E$
\end{center}
Therefore:
\begin{equation} \label{stress_strain}
\epsilon = \sigma E
\end{equation}

Thermal Stress:\\
If the system is statically indeterminate and it experiences a change in temperature, the resulting expansion will push against it's constraints and create compression. This compression is a reaction of the constraints. This reaction creates stress, or rather thermal stress. It is important to distinguish that thermal stress is not a component of thermal strain. We can quantify this by equating \ref{deform_thermal} and \ref{deform_load} because when it's statically indeterminate, $\delta = 0$:
\begin{center}
$\delta= \alpha \cdot \Delta T \cdot L = \frac{PL}{EA}$
\end{center}
The $L$ cancels and as a result:
\begin{center}
$\alpha \cdot \Delta T = \frac{P}{A} E$
\end{center}
Recall in equation \ref{stress} the relation for stress. By substituting $\frac{P}{A}$ with $\sigma$ and solving for it, we get:
\begin{equation} \label{thermal_stress}
\sigma_{thermal} = E \cdot  \alpha \cdot \Delta T
\end{equation}
Now that we have the necessary equations and relations, we may begin to analyze the structure.
\pagebreak

The first part of the problem is when $\Delta = 0.3 mm$
\newline

We will measure the total deformation of the beam by calculating the $\delta$ due to thermal expansion and applied loading.
\begin{equation}
\delta_{total} = \delta_{thermal} + \delta_{load}
\end{equation}

Calculation of thermal deformation using equation \ref{deform_thermal}:
\begin{gather*} \label{deform_thermal_calc_delta03}
\delta_{thermal} = \alpha \cdot \Delta T \cdot L \\
= \frac{10^{-5}}{C^{º}} \cdot 20^{º} \cdot 1000 mm \\
 = 0.2 mm
\end{gather*}

Calculation of deformation due to concentrated load using equation 
\begin{gather*} \label{deform_load_calc_delta03}
\delta_{load} = \frac{PL}{EA} \\
= \frac{10 kN \cdot 600 mm}{100 GPa \cdot (200 \cdot 200) mm^{2}} \\
=0.0015 mm
\end{gather*}

The total deformation is calculated using equation 8
\begin{gather*} 
\delta_{total} = \delta_{thermal} + \delta_{load}\\
= 0.2mm + 0.0015mm\\
=0.2015 mm
\end{gather*}

Because $\delta_{total}$ is less than $\Delta$, we can calculate strain using equation 2:
\begin{gather*}
\epsilon_{total} = \frac{\delta_{total}}{L_{º}}\\
=201.5 \cdot 10^{-6}
\end{gather*}

Total stress at $A$ can be found using equation 1
\begin{gather*}
\sigma = \frac{10 kN}{200 \cdot 200 mm^{2}} = 2.5 \cdot 10^{-4} GPa
\end{gather*}
\pagebreak

Next, we will analyze when $\Delta = 0.2mm$
\newline

$\delta_{total}$ was calculated to be 0.2015mm which is greater than 0.2mm. This loading condition will render it statically indeterminate. We will take the difference of 0.0015mm and calculate the strain using equation 2 like so:
\begin{gather} 
\epsilon^{'} = \frac{0.0015 mm}{1000mm} = 1.5 \cdot 10^{-6}
\end{gather}
However, because this body cannot deform any more due to it's constraints horizontally, we must consider the Poisson Effect: when a material is under compression, it will expand perpendicularly to the force. We can calculate this vertical strain using equation 5 as so:
\begin{gather}
0.25 = \frac{-\epsilon_{y}}{\epsilon_{x}}
\end{gather}

We set $\epsilon^{'} = -\epsilon_{x}$ in equation 10 to find the strain in y-component.
\begin{gather*}
(0.25)(1.5 \cdot 10^{-6}) = 375 \cdot 10^{-9}
\end{gather*}

To find the strain on the bar in the x-component can be calculated using equation 2, with $\delta = \Delta$ due to constraints.
\begin{gather} 
\epsilon_{x} = \frac{\Delta}{L_{º}} = 200 \cdot 10^{-6}
\end{gather}

In order to find the stress in point A, we must refer to equation 7.
\begin{gather*}
\sigma_{thermal} = E \cdot \alpha \cdot \Delta T 
= 100 GPa \cdot \frac{10^{-5}}{C^{º}} \cdot 20 ^{º}C\\
=0.2GPa
\end{gather*}
\pagebreak

Our last loading condition is when $\Delta = 0.1mm$
\newline

We will use the same approach as the last problem due to the deformation of the material being statically indeterminate. We will take the difference of $\Delta$ and $\delta$ and use it find the "negative" strain which in turn will allow us to find the strain in the vertical component using Poisson's Ratio:
\begin{gather} 
\delta_{imaginary} = \delta - \Delta = .1015\\
-\epsilon_{x} = \frac{.1015}{1000} = 101 \cdot 10^{-6}\\
\nu = \frac{\epsilon_{y}}{-\epsilon_{x}}
(0.25) \cdot (101 \cdot 10^{-6}) = \epsilon_{y} = 23.375 \cdot 10^{-6}
\end{gather}

To find strain in the horizontal we use $\Delta$ in equation 2:
\begin{gather*}
\epsilon_{x} = \frac{0.1}{1000} = 100 \cdot 10^{-6}
\end{gather*}







\end{document}